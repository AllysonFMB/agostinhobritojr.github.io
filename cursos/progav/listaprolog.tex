\documentclass[12pt,a4paper]{article}
\usepackage[latin1]{inputenc}
\usepackage{calc}
\usepackage{setspace}
\usepackage{graphicx}
\usepackage{multicol}
\usepackage[normalem]{ulem}
\usepackage[brazil]{babel}
\usepackage{color}
\usepackage{hyperref}
\usepackage{geometry}
\usepackage{listings}
\usepackage{sudoku}

\geometry{left=2.5cm, top=2.5cm, bottom=2.5cm, right=2cm}
 
\renewcommand{\lstlistingname}{Listagem}
\renewcommand{\labelenumii}{\alph{enumii})}
\newcommand{\lacuna}{(\hspace*{5mm}) }

\lstset{language=C, basicstyle=\ttfamily\small, showstringspaces=false, showspaces=false, keywordstyle=\color{black}\bfseries}

\begin{document}
\section*{DCA0201 - 1a. Lista de Exerc�cios}
\begin{enumerate}
  \item Crie uma regra para encontrar o �ltimo elemento de uma lista. Exemplo:
\begin{verbatim}
?- ultimo(X,[a,b,c,d]).
X = d
\end{verbatim}
  \item Crie uma regra para encontrar o pen�ltimo elemento de uma lista.
  \item Crie uma regra para encontrar o k-�simo elemento de uma lista.
  \item Crie uma regra para determinar se uma lista � um pal�ndromo. Pal�ndromos podem ser lidos para tr�s e para frente. Ex: $[s,o,p,a,p,o,s]$.
  \item Resolva o seguinte criptograma, sabendo que as letras podem assumir um dos algarismos de 0 a 9, todos diferentes.
    \begin{displaymath}
      \begin{array}{ccccc}
        & N & O & V & E \\
        + & T & R & E & S\\ \hline
        & D & O & Z & E\\
      \end{array}
    \end{displaymath}
  \item Crie uma base de regras para representar rela��es familiares, incluindo as rela��es de cunhado(a), concunhado(a) e tio(a)\_torto(a) (casado com a(o) tia(o) leg�timo)
  \item Considere uma representa��o de conjuntos como listas e defina os seguintes predicados:
    \begin{itemize}
    \item membro(X,L), que indica se um elemento X pertence ao conjunto L.
    \item subconjunto(L,K), que indica se o conjunto L � subconjunto de K.
    \item disjunto(L,K), que indica se L and K n�o possuem elementos em comum.
    \item uniao(L,K,M), que indica se is a uni�o de L e K.
    \item intersecao(L,K,M), que indica se M � a interse��o de L e K.
    \item differenca(L,K,M), que indica se M � a diferen�a de L e K.
    \end{itemize}
  \item Pesquise na Internet solu��es em prolog para realizar coloriza��o de maps geogr�ficos. O objetivo da coloriza��o � representar mapas cartogr�ficos com regi�es coloridas, de forma que duas regi�es adjacentes n�o tenham a mesma cor. Use os resultados da sua pesquisa para propor uma coloriza��o para o mapa do Brasil.
  \item Defina os predicados and/2, or/2, nand/2 e xor/2 que ir�o funcionar ou falhar de acordo com os resultados das suas opera��es. Espress�es l�gicas poder�o ser escritas da forma and(or(A,B),nand(A,B)).

Escreva um predicado tabela/3 que mostra a tabela verdade para uma dada espress�o l�gica de duas vari�veis. Exemplo:
\begin{verbatim}
?- tabela(A,B,and(A,or(A,B))).
true true true
true fail true
fail true fail
fail fail fail
\end{verbatim}
\item Defina um predicado ehprimo/1 para determinar se um n�mero � primo. Exemplo:
\begin{verbatim}
?- ehprimo(7).
true
\end{verbatim}

\end{enumerate}
\end{document}
