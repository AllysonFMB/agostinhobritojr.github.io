\documentclass[12pt,a4paper]{article}

\usepackage{epsfig}
\usepackage{multicol}
%\usepackage{noitemsep}
\usepackage[utf8]{inputenc}
\usepackage[portuges]{babel}
\usepackage{fancyheadings}
\usepackage{amsmath}
\usepackage{ulem}
\usepackage{subfigure}
\usepackage{tabularx}
\usepackage{geometry}
% As margens
\geometry{left=2.0cm, top=2.0cm, bottom=2.0cm, right=2cm}
 
\renewcommand{\labelenumii}{\alph{enumii})}
\newcommand{\lacuna}{(\hspace*{5mm}) }

\begin{document}
\begin{flushright}\scshape
  Universidade Federal do Rio Grande do Norte\\
  Centro de Tecnologia\\
  Departamento de Engenharia da Computação e Automação\\
  \rule{\linewidth}{1pt}
\end{flushright}

\begin{center}\Large
DCA0201 -- Paradigmas de programação (1\raisebox{0.5ex}{\small a} avaliação - 2016.2)\\
\end{center}
\noindent Aluno:\hrulefill \vspace{2mm} Matrícula: \rule{2.5cm}{0.4pt}

\begin{enumerate}
\item (2,0 pontos) Listando os seis primeiro números primos: 2, 3, 5, 7, 11 e 13,
  podemos ver que o sexto número primo é 13. Qual é o número primo de
  ordem 10001? Desenvolva um programa em linguagem Prolog para
  descobrir a resposta.
\item (2,0 pontos) Refaça o problema anterior utilizando a linguagem
  de programação funcional LISP.

\item (2,0 pontos) Observe o exemplo que segue.
\begin{verbatim}
#include <stdio.h>

void tac(void){ printf("tac\n"); }
void tec(void){ printf("tec\n"); }
void tic(void){ printf("tic\n"); }
void toc(void){ printf("toc\n"); }
void tuc(void){ printf("tuc\n"); }

int main(void){
  int i, j, k;
  void (*fp[5])(void) = {tac,tec,tic,toc,tuc};
  j = 237;
  for(i=1; i<=5; i++){
    k = (j >>= 1)%5;
    fp[k]();
  }
}
\end{verbatim}
  Compile este exemplo, execute-o, e explique o porquê de cada linha
  que aparece na saída, fundamentando adequadamente sua resposta.
\end{enumerate}
Instruções: ESCOLHA 2 QUESTÕES PARA RESPONDER E INDIQUE NO TOPO DO
RELATÓRIO DA PROVA QUAIS AS QUESTÕES QUE VOCÊ ESCOLHEU. Caso não sejam
indicadas as questões, assumir-se-á que foram escolhidas as duas
primeiras. Consulte a documentação que achar necessária {\bf apenas no
  computador}. Prepare um relatório em texto simples (usando emacs ou
gedit, por exemplo) conforme modelo apresentado no final desse
documento. Apresente os comentários que sejam necessários para o bom
entendimento da sua resposta, bem como as cópias dos códigos fontes no
dentro do proprio relatório.
\begin{verbatim}
=== MODELO DE RELATORIO DE PROVA EM TXT ===
Aluno:
Questoes escolhidas: 1 e 2
======================
questao 1

bla bla bla

======================
questao 2

bla bla bla

\end{verbatim}

\end{document}
